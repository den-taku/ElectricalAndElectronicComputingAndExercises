\documentclass[uplatex, 11pt,a4j, titlepage]{jsarticle}

\usepackage{assets/preamble}
\usepackage{assets/info}
\usepackage{listings,jlisting}

% \lstset{
% language = Rust,   
% breaklines = true,
% numbers = left,
% frame = tbrl,
% tabsize = 4,
% captionpos = t
% }

% Title
\title{第1回 演習課題}
\date{2020年 10月 29日}
\author{
    \small{\myid} \\
    \myname\thanks{\mymail}
}

\begin{document}
\maketitle

% 実験レポート1
% ここから

\subtitle{2020/10/8}

\theme{課題1.1}

二分法およびニュートン法を用いて非線形方程式を解くプログラムを
それぞれソースコード1、ソースコード2に示す。

\ 

まず二分法を用いたソースコード1について説明する。

bisection\_method 関数は、引数としてrange、e、fを受け取る。
これらはそれぞれ範囲、許容誤差、関数である。
まず初期区間をrangeとして与えると、bisection\_methodは
bisection\_method\_inner関数にrange、e、fを渡し、さらに
回数としてtimesに1を、反復回数上限値としてlimitに1,000,000を与える。
二分法は適切に初期区間を与えると必ず近似解が求まるためlimitはオプションである。
bisection\_method\_inner関数は範囲を半分に区切り、解が存在すると思われる範囲を
再帰的に渡してtimesを一つ進める。
この時その範囲が許容誤差内に収まったなら、半分に区切った時の値を近似解として返す。
ソースコード中に含まれるprintln!関数は、
課題で反復回数と誤差のcsvファイルを作成するために
標準出力に値を渡しているだけで求解に直接は影響しない。

\ 

次にニュートン法を用いたソースコード2について説明する。

まず、ニュートン法で非線形方程式を解くには関数を微分する必要がある。
関数の微分には、微分係数の定義である
\begin{equation}
    \lim_{h \to 0} \frac{f(x + h) - f(x)}{h}   
\end{equation}
を用いて微分した関数を返すdifferential\_f関数を作成した。

newton\_raphson\_method関数は次のようなアルゴリズムで方程式を解く。
まず、関数には$f(x)$と初期近似解を与える。すると、その関数を微分し、
\begin{equation}
    g(x) = x - \frac{f(x)}{f'(x)}
\end{equation}
となる$g(x)$を計算するnewton\_transform関数に$f(x)$、$f'(x)$を渡し、
また閾値と回数として1、反復回数上限として1,000,000とともに
newton\_method関数に渡す。

newton\_method関数では、$g(x)$を用いて近似解の候補を求め、元の$x$との
距離が閾値よりも小さい時、その計算した値を近似解として返す。閾値よりも大きかった場合は
計算した値を再帰的にnewton\_methodに渡す。それを繰り返すことで非線形方程式を解く。
最初にnextの値をチェックしているのは、$g(x)$の値が想定していない値になった時の処理を
まとめてあるだけであり、アルゴリズムに直接は影響しない。これについては後で言及する。

ここでも後で値をplotするためにprintln!関数を用いて標準出力に値を渡している。


\section{課題1.1.1}
\begin{equation}
    f(x) = x^5 - 3 x^4 + x^3 + 5 x^2 - 6x + 2 
\end{equation}

ここで、

二分法の初期期間を[-2, 0]とし、ニュートン法の初期近似解を-1とする。
そしてそれぞれの手法で得られた近似解と

\begin{lstlisting}
    fn main() {
        println!("Hello, World!");
    }
\end{lstlisting}

% \section{ソースコードの表示}
% \begin{lstlisting}
%     //
% \end{lstlisting}
\section{実験}
\section{実験結果}
\section{考察}
\section{課題}

% 実験レポート1
% ここまで

\newpage
\resetcounters

% 実験レポート2
% ここから

\subtitle{2020/10/*}

\theme{実験2:タイトル}
\section{目的}
\section{実験}
\section{実験結果}
\section{考察}
\section{課題}

% 実験レポート2
% ここまで

\newpage
\resetcounters


% 実験レポート3
% ここから

\subtitle{2019/*/*}

\theme{実験3:タイトル}
\section{目的}
\section{実験}
\section{実験結果}
\section{考察}
\section{課題}

% 実験レポート3
% ここまで

% 参考文献
\newpage
\thispagestyle{empty}
\nocite{key1}
\nocite{key2}
\bibliographystyle{junsrt}
\bibliography{assets/ref}

\end{document}