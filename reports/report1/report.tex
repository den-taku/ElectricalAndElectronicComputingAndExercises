\documentclass[uplatex, 11pt,a4j, titlepage]{jsarticle}

\usepackage{assets/preamble}
\usepackage{assets/info}
\usepackage{listings,jlisting}

% \lstset{
% language = Rust,   
% breaklines = true,
% numbers = left,
% frame = tbrl,
% tabsize = 4,
% captionpos = t
% }

% Title
\title{第1回 演習課題}
\date{2020年 10月 29日}
\author{
    \small{\myid} \\
    \myname\thanks{\mymail}
}

\begin{document}
\maketitle

% 実験レポート1
% ここから

\subtitle{2020/10/8}

\theme{課題1.1}

二分法およびニュートン法を用いて非線形方程式を解くプログラムを
それぞれソースコード1、ソースコード2に示す。

\ 

まず二分法を用いたソースコード1について説明する。

bisection\_method 関数は、引数としてrange、e、fを受け取る。
これらはそれぞれ範囲、許容誤差、関数である。
まず初期区間をrangeとして与えると、bisection\_methodは
bisection\_method\_inner関数にrange、e、fを渡し、さらに
回数としてtimesに1を、反復回数上限値としてlimitに1,000,000を与える。
二分法は適切に初期区間を与えると必ず近似解が求まるためlimitはオプションである。
bisection\_method\_inner関数は範囲を半分に区切り、解が存在すると思われる範囲を
再帰的に渡してtimesを一つ進める。
この時その範囲が許容誤差内に収まったなら、半分に区切った時の値を近似解として返す。
ソースコード中に含まれるprintln!関数は、
課題で反復回数と誤差のcsvファイルを作成するために
標準出力に値を渡しているだけで求解に直接は影響しない。

\ 

次にニュートン法を用いたソースコード2について説明する。

まず、ニュートン法で非線形方程式を解くには関数を微分する必要がある。
関数の微分には、微分係数の定義である
\begin{equation}
    \lim_{h \to 0} \frac{f(x + h) - f(x)}{h}   
\end{equation}
を用いて微分した関数を返すdifferential\_f関数を作成した。

newton\_raphson\_method関数は次のようなアルゴリズムで方程式を解く。
まず、関数には$f(x)$と初期近似解を与える。すると、その関数を微分し、
\begin{equation}
    g(x) = x - \frac{f(x)}{f'(x)}
\end{equation}
となる$g(x)$を計算するnewton\_transform関数に$f(x)$、$f'(x)$を渡し、
また閾値と回数として1、反復回数上限として1,000,000とともに
newton\_method関数に渡す。

newton\_method関数では、$g(x)$を用いて近似解の候補を求め、元の$x$との
距離が閾値よりも小さい時、その計算した値を近似解として返す。閾値よりも大きかった場合は
計算した値を再帰的にnewton\_methodに渡す。それを繰り返すことで非線形方程式を解く。
最初にnextの値をチェックしているのは、$g(x)$の値が想定していない値になった時の処理を
まとめてあるだけであり、アルゴリズムに直接は影響しない。これについては後で言及する。

ここでも後で値をplotするためにprintln!関数を用いて標準出力に値を渡している。

\ 

\begin{lstlisting}[caption={bisection\_method.rs}]

use std::ops::Range;
use std::rc::Rc;
     
fn bisection_method(
    mut range: Range<f64>, 
    e: f64, 
    f: Rc<dyn Fn(f64) -> f64>
)-> f64 {
    bisection_method_inner(range, e, f, 1, 1000000)
}
     
fn bisection_method_inner(
    mut range: Range<f64>,
    e: f64,
    f: Rc<dyn Fn(f64) -> f64>,
    times: usize,
    limit: usize,
) -> f64 {
    let x_new = (range.end + range.start) / 2.;
    if times == limit {
        return x_new;
    }
    if f(x_new) * f(range.start) >= 0. {
        range.start = x_new;
    } else {
        range.end = x_new;
    }
    println!("{}, {}", times, (x_new - 1.414213566237).abs()); 
    if range.end - range.start <= e {
        x_new
    } else {
        bisection_method_inner(range, e, f, times + 1, limit)
    }
}

\end{lstlisting}

\begin{lstlisting}[caption={newton\_raphson\_method.rs}]
use std::rc::Rc;
use std::result::Result;
         
fn newton_raphson_method(
    f: Rc<dyn Fn(f64) -> f64>, 
    init: f64
) -> Result<f64, String> {
    let threshold = 0.1e-10;
    let f_dir = differential_f(f.clone()); 
    newton_method(newton_transform(f, f_dir), 
                  init, 
                  threshold, 
                  1, 
                  1000000)
}
        
fn differential_f(f: Rc<dyn Fn(f64) -> f64>) 
    -> Rc<dyn Fn(f64) -> f64> {
    let dx = 0.1e-10;
    let f_dir = move |x: f64| 
                    -> f64 { (f(x + dx) - f(x)) / dx };
    Rc::new(f_dir)
}
        
fn newton_transform(
    f: Rc<dyn Fn(f64) -> f64>,
    f_dir: Rc<dyn Fn(f64) -> f64>,
) -> Rc<dyn Fn(f64) -> f64> {
    Rc::new(move |x: f64| -> f64 { x - f(x) / f_dir(x) })
}
        
fn newton_method(
    f: Rc<dyn Fn(f64) -> f64>,
    guess: f64,
    threshold: f64,
    times: usize,
    limit: usize,
) -> Result<f64, String> {
    let next = f(guess);
    if next == f64::NEG_INFINITY 
    || next == f64::INFINITY 
    || next.is_nan() {
        return Err(
            format!(
                "x^(k+1) is not a number: last value is {}.", 
                guess
        ));
    }
    if limit == times + 1 {
        return Err(format!(
            "solution doesn't converge: last value is {}.",
            next
        ));
    }
    if (next - guess).abs() <= threshold {
        Ok(next)
    } else {
        println!("{}, {},", times, (next - 1.414213566237).abs());
        newton_method(f, next, threshold, times + 1, limit)
    }
} 
\end{lstlisting}


\section{課題1.1.1}
\begin{equation}
    f(x) = x^5 - 3 x^4 + x^3 + 5 x^2 - 6x + 2 
\end{equation}
とする。
最初に説明した二分法およびニュートン法を用いて非線形方程式を解くプログラムを用いて課題を解く。


二分法の初期期間を[-2, 0]とし、ニュートン法の初期近似解を-1とする。
そして反復回数を横軸に、
それぞれの手法で得られた近似解と真値($\sqrt{2}$)との誤差の絶対値を縦軸にとった
片対数グラフをそれぞれ図1、図2に作成し、示す。



% \section{ソースコードの表示}
% \begin{lstlisting}
%     //
% \end{lstlisting}
\section{実験}
\section{実験結果}
\section{考察}
\section{課題}

% 実験レポート1
% ここまで

\newpage
\resetcounters

% 実験レポート2
% ここから

\subtitle{2020/10/*}

\theme{実験2:タイトル}
\section{目的}
\section{実験}
\section{実験結果}
\section{考察}
\section{課題}

% 実験レポート2
% ここまで

\newpage
\resetcounters


% 実験レポート3
% ここから

\subtitle{2019/*/*}

\theme{実験3:タイトル}
\section{目的}
\section{実験}
\section{実験結果}
\section{考察}
\section{課題}

% 実験レポート3
% ここまで

% 参考文献
\newpage
\thispagestyle{empty}
\nocite{key1}
\nocite{key2}
\bibliographystyle{junsrt}
\bibliography{assets/ref}

\end{document}